\documentclass[]{article}
\usepackage{amsfonts, amssymb, amsmath}
\usepackage{float}
\usepackage{graphicx}

\title{Gate 64.2022}
\author{Umair Parwez \\ EE22BTECH11009}
\date{}
\begin{document}
\maketitle
\providecommand{\pr}[1]{\ensuremath{\Pr\left(#1\right)}}
\providecommand{\prt}[2]{\ensuremath{p_{#1}^{\left(#2\right)} }}        % own macro for this question
\providecommand{\qfunc}[1]{\ensuremath{Q\left(#1\right)}}
\providecommand{\sbrak}[1]{\ensuremath{{}\left[#1\right]}}
\providecommand{\lsbrak}[1]{\ensuremath{{}\left[#1\right.}}
\providecommand{\rsbrak}[1]{\ensuremath{{}\left.#1\right]}}
\providecommand{\brak}[1]{\ensuremath{\left(#1\right)}}
\providecommand{\lbrak}[1]{\ensuremath{\left(#1\right.}}
\providecommand{\rbrak}[1]{\ensuremath{\left.#1\right)}}
\providecommand{\cbrak}[1]{\ensuremath{\left\{#1\right\}}}
\providecommand{\lcbrak}[1]{\ensuremath{\left\{#1\right.}}
\providecommand{\rcbrak}[1]{\ensuremath{\left.#1\right\}}}
\newcommand{\sgn}{\mathop{\mathrm{sgn}}}
\providecommand{\abs}[1]{\left\vert#1\right\vert}
\providecommand{\res}[1]{\Res\displaylimits_{#1}} 
\providecommand{\norm}[1]{\left\lVert#1\right\rVert}
%\providecommand{\norm}[1]{\lVert#1\rVert}
\providecommand{\mtx}[1]{\mathbf{#1}}
\providecommand{\mean}[1]{E\left[ #1 \right]}
\providecommand{\cond}[2]{#1\middle|#2}
\providecommand{\fourier}{\overset{\mathcal{F}}{ \rightleftharpoons}}
\newenvironment{amatrix}[1]{%
  \left(\begin{array}{@{}*{#1}{c}|c@{}}
}{%
  \end{array}\right)
}
%\providecommand{\hilbert}{\overset{\mathcal{H}}{ \rightleftharpoons}}
%\providecommand{\system}{\overset{\mathcal{H}}{ \longleftrightarrow}}
	%\newcommand{\solution}[2]{\textbf{Solution:}{#1}}
\newcommand{\solution}{\noindent \textbf{Solution: }}
\newcommand{\cosec}{\,\text{cosec}\,}
\providecommand{\dec}[2]{\ensuremath{\overset{#1}{\underset{#2}{\gtrless}}}}
\newcommand{\myvec}[1]{\ensuremath{\begin{pmatrix}#1\end{pmatrix}}}
\newcommand{\mydet}[1]{\ensuremath{\begin{vmatrix}#1\end{vmatrix}}}
\newcommand{\myaugvec}[2]{\ensuremath{\begin{amatrix}{#1}#2\end{amatrix}}}
\providecommand{\rank}{\text{rank}}
\providecommand{\pr}[1]{\ensuremath{\Pr\left(#1\right)}}
\providecommand{\qfunc}[1]{\ensuremath{Q\left(#1\right)}}
	\newcommand*{\permcomb}[4][0mu]{{{}^{#3}\mkern#1#2_{#4}}}
\newcommand*{\perm}[1][-3mu]{\permcomb[#1]{P}}
\newcommand*{\comb}[1][-1mu]{\permcomb[#1]{C}}
\providecommand{\qfunc}[1]{\ensuremath{Q\left(#1\right)}}
\providecommand{\gauss}[2]{\mathcal{N}\ensuremath{\left(#1,#2\right)}}
\providecommand{\diff}[2]{\ensuremath{\frac{d{#1}}{d{#2}}}}
\providecommand{\myceil}[1]{\left \lceil #1 \right \rceil }
\newcommand\figref{Fig.~\ref}
\newcommand\tabref{Table~\ref}
\newcommand{\sinc}{\,\text{sinc}\,}
\newcommand{\rect}{\,\text{rect}\,}
%%
%	%\newcommand{\solution}[2]{\textbf{Solution:}{#1}}
%\newcommand{\solution}{\noindent \textbf{Solution: }}
%\newcommand{\cosec}{\,\text{cosec}\,}
%\numberwithin{equation}{section}
%\numberwithin{equation}{subsection}
%\numberwithin{problem}{section}
%\numberwithin{definition}{section}
%\makeatletter
%\@addtoreset{figure}{problem}
%\makeatother

%\let\StandardTheFigure\thefigure
\let\vec\mathbf

Consider a real valued source whose samples are independent and identically
distributed random variables with the probability density function, $f(x)$, as shown in
the figure.

\begin{figure}[h]
    \centering
    \includegraphics[width=9cm]{./figs/figure1.png}
    \caption{}
    \label{fig:15st/2023}    
\end{figure}

Consider a 1 bit quantizer that maps positive samples to value $\alpha$ and others to value
$\beta$. If $\alpha ^*$ and $\beta ^*$ are the respective choices for $\alpha$ and $\beta$ that minimize the mean square
quantization error, then find ($\alpha ^* - \beta ^*$)

\solution
First we must find value of A. Area under the PDF graph should be equal to 1. Thus, we can say,

\begin{align}
    (\frac{1}{2}\cdot A\cdot 2) + (1\cdot A) = 1 \\
    A = 0.5
\end{align}

Using the value of A, we can write $f_X(x)$ as a piecwise function given by,

\begin{align}
    f_X(x) = 
    \begin{cases}
        0.25x + 0.5 & -2\leq x\leq 0 \\
        0.5 & 0\leq x \leq1
    \end{cases}
\end{align}

Now, if $X_q$ be the output of the quantizer, then,

\begin{align}
    X_q = 
    \begin{cases}
        \alpha & 0\leq x \leq1 \\
        \beta & -2\leq x\leq 0
    \end{cases}
\end{align}

Mean square quantization error, 
\begin{align}
    MSQ(Q_e) &= E(Q_e^2) \\
    &= E((X-X_q)^2) \\
    &= \int_{-\infty}^{\infty}(x-X_q)^2\cdot f_X(x)\cdot dx \\
    &= \int_{-2}^{0}(x-\beta)^2\cdot f_X(x)\cdot dx + \int_{0}^{1}(x-\alpha)^2\cdot f_X(x)\cdot dx \\
    &= \int_{-2}^{0}(x^2+\beta ^2-2x\beta)\cdot (0.25x+0.5)\cdot dx \\ &\qquad + \int_{0}^{1}(x-\alpha)^2\cdot 0.5\cdot dx \\
    &= \frac{\beta ^2}{2} + \frac{2\beta}{3} -\frac{1}{3} + \frac{1}{6}[(1-\alpha)^3+\alpha ^3]
\end{align}

Now, to minimize $MSQ(Q_e)$,
\begin{align}
    \frac{\partial}{\partial \beta}\cdot MSQ(Q_e) = 0 \label{partbeta} \\
    \frac{\partial}{\partial \alpha}\cdot MSQ(Q_e) = 0 \label{partalpha}
\end{align}

\begin{enumerate}
    \item First we will consider \eqref{partbeta},
        \begin{align}
            \frac{\partial}{\partial \beta}\cdot MSQ(Q_e) = 0 \\
            \beta + \frac{2}{3} =0 \\
            \beta = -\frac{2}{3}
        \end{align}
    \item Now we will consider \eqref{partalpha},
        \begin{align}
            \frac{\partial}{\partial \alpha}\cdot MSQ(Q_e) = 0 \\
            \frac{1}{6}[3(1-\alpha)^2(-1) + 3\alpha ^2] = 0 \\
            \alpha = \frac{1}{2}
        \end{align}
\end{enumerate}

Therefore the values of $\alpha$ and $\beta$ for which the mean square error is minimum are,
\begin{align}
    \alpha ^* &= \frac{1}{2} \\
    \beta ^* &= -\frac{2}{3}
\end{align}

Thus, 
\begin{align}
    (\alpha ^* - \beta ^*) = \frac{7}{6}
\end{align}



\end{document}